\documentclass[]{article}
\usepackage{lmodern}
\usepackage{amssymb,amsmath}
\usepackage{ifxetex,ifluatex}
\usepackage{fixltx2e} % provides \textsubscript
\ifnum 0\ifxetex 1\fi\ifluatex 1\fi=0 % if pdftex
  \usepackage[T1]{fontenc}
  \usepackage[utf8]{inputenc}
\else % if luatex or xelatex
  \ifxetex
    \usepackage{mathspec}
  \else
    \usepackage{fontspec}
  \fi
  \defaultfontfeatures{Ligatures=TeX,Scale=MatchLowercase}
\fi
% use upquote if available, for straight quotes in verbatim environments
\IfFileExists{upquote.sty}{\usepackage{upquote}}{}
% use microtype if available
\IfFileExists{microtype.sty}{%
\usepackage{microtype}
\UseMicrotypeSet[protrusion]{basicmath} % disable protrusion for tt fonts
}{}
\usepackage[margin=1in]{geometry}
\usepackage{hyperref}
\hypersetup{unicode=true,
            pdftitle={Lab5},
            pdfauthor={Me},
            pdfborder={0 0 0},
            breaklinks=true}
\urlstyle{same}  % don't use monospace font for urls
\usepackage{color}
\usepackage{fancyvrb}
\newcommand{\VerbBar}{|}
\newcommand{\VERB}{\Verb[commandchars=\\\{\}]}
\DefineVerbatimEnvironment{Highlighting}{Verbatim}{commandchars=\\\{\}}
% Add ',fontsize=\small' for more characters per line
\usepackage{framed}
\definecolor{shadecolor}{RGB}{248,248,248}
\newenvironment{Shaded}{\begin{snugshade}}{\end{snugshade}}
\newcommand{\KeywordTok}[1]{\textcolor[rgb]{0.13,0.29,0.53}{\textbf{{#1}}}}
\newcommand{\DataTypeTok}[1]{\textcolor[rgb]{0.13,0.29,0.53}{{#1}}}
\newcommand{\DecValTok}[1]{\textcolor[rgb]{0.00,0.00,0.81}{{#1}}}
\newcommand{\BaseNTok}[1]{\textcolor[rgb]{0.00,0.00,0.81}{{#1}}}
\newcommand{\FloatTok}[1]{\textcolor[rgb]{0.00,0.00,0.81}{{#1}}}
\newcommand{\ConstantTok}[1]{\textcolor[rgb]{0.00,0.00,0.00}{{#1}}}
\newcommand{\CharTok}[1]{\textcolor[rgb]{0.31,0.60,0.02}{{#1}}}
\newcommand{\SpecialCharTok}[1]{\textcolor[rgb]{0.00,0.00,0.00}{{#1}}}
\newcommand{\StringTok}[1]{\textcolor[rgb]{0.31,0.60,0.02}{{#1}}}
\newcommand{\VerbatimStringTok}[1]{\textcolor[rgb]{0.31,0.60,0.02}{{#1}}}
\newcommand{\SpecialStringTok}[1]{\textcolor[rgb]{0.31,0.60,0.02}{{#1}}}
\newcommand{\ImportTok}[1]{{#1}}
\newcommand{\CommentTok}[1]{\textcolor[rgb]{0.56,0.35,0.01}{\textit{{#1}}}}
\newcommand{\DocumentationTok}[1]{\textcolor[rgb]{0.56,0.35,0.01}{\textbf{\textit{{#1}}}}}
\newcommand{\AnnotationTok}[1]{\textcolor[rgb]{0.56,0.35,0.01}{\textbf{\textit{{#1}}}}}
\newcommand{\CommentVarTok}[1]{\textcolor[rgb]{0.56,0.35,0.01}{\textbf{\textit{{#1}}}}}
\newcommand{\OtherTok}[1]{\textcolor[rgb]{0.56,0.35,0.01}{{#1}}}
\newcommand{\FunctionTok}[1]{\textcolor[rgb]{0.00,0.00,0.00}{{#1}}}
\newcommand{\VariableTok}[1]{\textcolor[rgb]{0.00,0.00,0.00}{{#1}}}
\newcommand{\ControlFlowTok}[1]{\textcolor[rgb]{0.13,0.29,0.53}{\textbf{{#1}}}}
\newcommand{\OperatorTok}[1]{\textcolor[rgb]{0.81,0.36,0.00}{\textbf{{#1}}}}
\newcommand{\BuiltInTok}[1]{{#1}}
\newcommand{\ExtensionTok}[1]{{#1}}
\newcommand{\PreprocessorTok}[1]{\textcolor[rgb]{0.56,0.35,0.01}{\textit{{#1}}}}
\newcommand{\AttributeTok}[1]{\textcolor[rgb]{0.77,0.63,0.00}{{#1}}}
\newcommand{\RegionMarkerTok}[1]{{#1}}
\newcommand{\InformationTok}[1]{\textcolor[rgb]{0.56,0.35,0.01}{\textbf{\textit{{#1}}}}}
\newcommand{\WarningTok}[1]{\textcolor[rgb]{0.56,0.35,0.01}{\textbf{\textit{{#1}}}}}
\newcommand{\AlertTok}[1]{\textcolor[rgb]{0.94,0.16,0.16}{{#1}}}
\newcommand{\ErrorTok}[1]{\textcolor[rgb]{0.64,0.00,0.00}{\textbf{{#1}}}}
\newcommand{\NormalTok}[1]{{#1}}
\usepackage{graphicx,grffile}
\makeatletter
\def\maxwidth{\ifdim\Gin@nat@width>\linewidth\linewidth\else\Gin@nat@width\fi}
\def\maxheight{\ifdim\Gin@nat@height>\textheight\textheight\else\Gin@nat@height\fi}
\makeatother
% Scale images if necessary, so that they will not overflow the page
% margins by default, and it is still possible to overwrite the defaults
% using explicit options in \includegraphics[width, height, ...]{}
\setkeys{Gin}{width=\maxwidth,height=\maxheight,keepaspectratio}
\IfFileExists{parskip.sty}{%
\usepackage{parskip}
}{% else
\setlength{\parindent}{0pt}
\setlength{\parskip}{6pt plus 2pt minus 1pt}
}
\setlength{\emergencystretch}{3em}  % prevent overfull lines
\providecommand{\tightlist}{%
  \setlength{\itemsep}{0pt}\setlength{\parskip}{0pt}}
\setcounter{secnumdepth}{0}
% Redefines (sub)paragraphs to behave more like sections
\ifx\paragraph\undefined\else
\let\oldparagraph\paragraph
\renewcommand{\paragraph}[1]{\oldparagraph{#1}\mbox{}}
\fi
\ifx\subparagraph\undefined\else
\let\oldsubparagraph\subparagraph
\renewcommand{\subparagraph}[1]{\oldsubparagraph{#1}\mbox{}}
\fi

%%% Use protect on footnotes to avoid problems with footnotes in titles
\let\rmarkdownfootnote\footnote%
\def\footnote{\protect\rmarkdownfootnote}

%%% Change title format to be more compact
\usepackage{titling}

% Create subtitle command for use in maketitle
\newcommand{\subtitle}[1]{
  \posttitle{
    \begin{center}\large#1\end{center}
    }
}

\setlength{\droptitle}{-2em}

  \title{Lab5}
    \pretitle{\vspace{\droptitle}\centering\huge}
  \posttitle{\par}
    \author{Me}
    \preauthor{\centering\large\emph}
  \postauthor{\par}
      \predate{\centering\large\emph}
  \postdate{\par}
    \date{16 grudnia 2018}


\begin{document}
\maketitle

\begin{Shaded}
\begin{Highlighting}[]
\CommentTok{%metoda Jacobiego}
\FunctionTok{clc}
\FunctionTok{clear} \FunctionTok{all}
\FunctionTok{tic}

\CommentTok{%funkcja}
\NormalTok{F = @(x,y) -}\FunctionTok{cos}\NormalTok{(x+y)-}\FunctionTok{cos}\NormalTok{(x-y);}

\CommentTok{%rozwi?zanie analityczne}
\NormalTok{G = @(x,y) }\FunctionTok{cos}\NormalTok{(x).*}\FunctionTok{cos}\NormalTok{(y);}

\CommentTok{%przedzia? omega}
\NormalTok{xa=}\FloatTok{0}\NormalTok{;}
\NormalTok{xb=}\BaseNTok{pi}\NormalTok{;}
\NormalTok{yc=}\FloatTok{0}\NormalTok{;}
\NormalTok{yd=}\BaseNTok{pi}\NormalTok{/}\FloatTok{2}\NormalTok{;}

\CommentTok{%warunki brzegowe}
\NormalTok{u1 = @(x) }\FunctionTok{cos}\NormalTok{(x);}
\NormalTok{u2 = @(y) -}\FunctionTok{cos}\NormalTok{(y);}
\NormalTok{u3 = @(x) }\FloatTok{0}\NormalTok{;}
\NormalTok{u4 = @(y) }\FunctionTok{cos}\NormalTok{(y);}

\CommentTok{%siatka}
\NormalTok{n=}\FloatTok{5}\NormalTok{;}

\NormalTok{h=(xb-xa)/(n+}\FloatTok{1}\NormalTok{);}
\NormalTok{k=(yd-yc)/(xb-xa)*(n+}\FloatTok{1}\NormalTok{)-}\FloatTok{1}\NormalTok{;}
\NormalTok{x=[xa:h:xb];}
\NormalTok{y=[yc:h:yd];}

\NormalTok{tol=}\FloatTok{1e-4}\NormalTok{;}
\FunctionTok{error} \NormalTok{= }\FloatTok{1}\NormalTok{; }
\NormalTok{licznik=}\FloatTok{0}\NormalTok{; }

\CommentTok{%tworzenie macierzy}
\NormalTok{U1(}\FloatTok{1}\NormalTok{:n+}\FloatTok{2}\NormalTok{) = u1(x);}
\NormalTok{U2(}\FloatTok{1}\NormalTok{:k+}\FloatTok{2}\NormalTok{) = u2(y(}\FloatTok{1}\NormalTok{:(k+}\FloatTok{2}\NormalTok{)));}
\NormalTok{U3(}\FloatTok{1}\NormalTok{:n+}\FloatTok{2}\NormalTok{) = u3(x);}
\NormalTok{U4(}\FloatTok{1}\NormalTok{:k+}\FloatTok{2}\NormalTok{) = u4(y(}\FloatTok{1}\NormalTok{:(k+}\FloatTok{2}\NormalTok{)));}

\NormalTok{U(}\FloatTok{1}\NormalTok{,:) = U1(}\FloatTok{1}\NormalTok{:n+}\FloatTok{2}\NormalTok{);}
\NormalTok{U(k+}\FloatTok{2}\NormalTok{,:) = U3(}\FloatTok{1}\NormalTok{:n+}\FloatTok{2}\NormalTok{);}
\NormalTok{U(:,}\FloatTok{1}\NormalTok{) = U4(}\FloatTok{1}\NormalTok{:k+}\FloatTok{2}\NormalTok{);}
\NormalTok{U(:,n+}\FloatTok{2}\NormalTok{) = U2(}\FloatTok{1}\NormalTok{:k+}\FloatTok{2}\NormalTok{);}

\NormalTok{Uk=U;}

\NormalTok{while }\FunctionTok{error}\NormalTok{>tol}
    \NormalTok{licznik = licznik+}\FloatTok{1}\NormalTok{;}
    \NormalTok{for }\BaseNTok{i}\NormalTok{=}\FloatTok{2}\NormalTok{:k+}\FloatTok{1}
        \NormalTok{for }\BaseNTok{j}\NormalTok{=}\FloatTok{2}\NormalTok{:n+}\FloatTok{1}
            \NormalTok{Uk(}\BaseNTok{i}\NormalTok{,}\BaseNTok{j}\NormalTok{) =}\FloatTok{0.25}\NormalTok{*(U(}\BaseNTok{i}\NormalTok{+}\FloatTok{1}\NormalTok{,}\BaseNTok{j}\NormalTok{)+U(}\BaseNTok{i}\NormalTok{-}\FloatTok{1}\NormalTok{,}\BaseNTok{j}\NormalTok{)+U(}\BaseNTok{i}\NormalTok{,}\BaseNTok{j}\NormalTok{+}\FloatTok{1}\NormalTok{)+U(}\BaseNTok{i}\NormalTok{,}\BaseNTok{j}\NormalTok{-}\FloatTok{1}\NormalTok{))-}\FloatTok{0.25}\NormalTok{*h^}\FloatTok{2}\NormalTok{*F(x(}\BaseNTok{j}\NormalTok{),y(}\BaseNTok{i}\NormalTok{));}
        \NormalTok{end}
    \NormalTok{end}
    
    \FunctionTok{error} \NormalTok{= }\FunctionTok{max}\NormalTok{(}\FunctionTok{max}\NormalTok{(}\FunctionTok{abs}\NormalTok{(Uk-U)));}

    \NormalTok{U=Uk;}
\NormalTok{end}

\CommentTok{%wykresy}
\NormalTok{[X,Y] = }\FunctionTok{meshgrid}\NormalTok{(x,y);}
\FunctionTok{subplot}\NormalTok{(}\FloatTok{1}\NormalTok{,}\FloatTok{2}\NormalTok{,}\FloatTok{1}\NormalTok{)}
\FunctionTok{surf}\NormalTok{(X,Y,U)}
\FunctionTok{title}\NormalTok{(}\StringTok{'Metoda Numeryczna'}\NormalTok{)}
\FunctionTok{subplot}\NormalTok{(}\FloatTok{1}\NormalTok{,}\FloatTok{2}\NormalTok{,}\FloatTok{2}\NormalTok{)}
\FunctionTok{surf}\NormalTok{(X,Y,(G(X,Y)))}
\FunctionTok{title}\NormalTok{(}\StringTok{'Metoda Analityczna'}\NormalTok{)}

\NormalTok{blad = }\FunctionTok{max}\NormalTok{(}\FunctionTok{max}\NormalTok{(}\FunctionTok{abs}\NormalTok{(U-G(X,Y))))}
\NormalTok{licznik}
\FunctionTok{toc}
\end{Highlighting}
\end{Shaded}

\begin{verbatim}
## blad =  0.0036873
## licznik =  13
## Elapsed time is 0.230382 seconds.
\end{verbatim}


\end{document}
