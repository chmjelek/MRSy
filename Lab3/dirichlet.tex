\section{Zagadnienie brzegowe Dirichleta}
\subsection{Definicja}

Warunek brzegowy Dirichleta stosujemy w teorii równań różniczkowych zwyczajnych oraz cząstkowych (typu eliptycznego) II rzędu.

Polega ona na zalożeniu, że znane są wartości poszukiwanego rozwiązania na obu brzegach przedziału $<a,b>$, na którym określona jest funkcja będąca rozwiązaniem danego problemu.

Jeżeli dla równania różniczkowego (zwyczajnego lub cząstkowego) stawiamy warunek brzegowy Dirichleta (na całym brzegu), to mówimy o zagadnieniu (problemie) Dirichleta. 

\textbf{Warunek brzegowy Dirichleta (I rodzaju)}

\[
\begin{cases}
\vspace{0.1cm} 
\hspace{0,1cm}\alpha \cdot y'' + \beta \cdot y'+\gamma \cdot y=f \\
\vspace{0.1cm}
\hspace{0,1cm}y(a)=y_{a} \\
\hspace{0,1cm}y(b)=y_{b}
\end{cases}
\]
, gdzie:

$\alpha, \beta, \gamma, f$ są znanymi funkcjami
\newline
$y = y(x)$ jest poszukiwanym rozwiązaniem
\newline

\textbf{Zagadnienie Dirichleta}

Poszukujemy funckję $u$, której znane są wartości na brzegu. Taki problem da się rozwiązać analitycznie całkując dwukrotnie równanie.

\[
\begin{cases}
\vspace{0.1cm} 
\hspace{0,1cm}u''=f \\
\vspace{0.1cm}
\hspace{0,1cm}u|_{x=a}=u_{a} \\
\hspace{0,1cm}u|_{x=b}=u_{b}
\end{cases}
\]
\newline

My jednak sformułujemy to zagadnienie z użyciem MRS, otrzymamy wówczas aproksymację rozwiązania w wybranych przez nas węzłach.

Następnie porównamy metodę numeryczną oraz analityczną rozwiązania zadanego problemu.

\subsection{Cel ćwiczenia}

Naszym zadaniem było stworzenie algorytmu rozwiązującego następujące zagadnienia Dirichleta:

a)
\[
\begin{cases}
\vspace{0.1cm} 
\hspace{0,1cm}u''=-sin(x)-4\cdot sin(2x) \\
\vspace{0.1cm}
\hspace{0,1cm}u|_{x=0}=0 \\
\hspace{0,1cm}u|_{x=2\pi}=0
\end{cases}
\]
, gdzie:

$x\in [0,2\pi]$

Rozwiązanie analityczne: $\overline{\rm u}(x) = sin(x) +sin(2x)$
\newpage
b)
\[
\begin{cases}
\vspace{0.1cm} 
\hspace{0,1cm}u''=12x \\
\vspace{0.1cm}
\hspace{0,1cm}u|_{x=0}=0 \\
\hspace{0,1cm}u|_{x=1}=1
\end{cases}
\]
, gdzie:

$x\in [0,1]$

Rozwiązanie analityczne: $\overline{\rm u}(x) = 2x^{3}-x$

\vspace{0.3cm}
Ponadto zaprezentujemy wykres porównujący rozwiązanie numeryczne z rozwiązaniem analitycznym, a także wykres błędu $||E||_{\infty}$ w zależności od liczby obranych węzłów (n).

\subsection{Algorytm}

\subsection{Wykres}
 

